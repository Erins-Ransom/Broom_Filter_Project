\title{A Survey of Adaptive Filter Techniques}
\author{
        Eric Knorr, Richard Xu, Zachary Yedidia \\ % alphabetical order by last name
        \\
        CS 223\\
}
\date{\today}

\documentclass[11pt]{article}
\usepackage{amsmath}
\usepackage{graphicx}
\usepackage{caption}
\usepackage{subcaption}
\usepackage{placeins}
\usepackage[margin=1in]{geometry}
\usepackage{float}
\usepackage{wrapfig}
\usepackage{subfiles}


\begin{document}
\maketitle

\begin{abstract}
        Bloom filters and other approximate membership query data structures, or
    AMQs, are often used to help avoid costly lookups to large dictionaries.
    They achieve this by keeping a small, probabilistic representation of a set
    {\bf S} of keys from a universe {\bf U}.  All AMQs support inserts and
    lookups while some also support deletes.  When performing a lookup for an
    item $x$ that has been inserted, i.e. $x\in {\bf S}$, the AMQ will always
    return {\it present}.  The downside of their compact representation of {\bf S}
    is that, when a lookup is performed for an item $x \notin {\bf S}$, {\it
    present} may be returned with probability $\epsilon$.  This {\bf
    false-positive probability} is a usually a tunable parameter with a lower
    $\epsilon$ typically requiring more space to be used by the AMQ.  That
    said, their compact representation is also what makes them helpful since
    they can be kept in memory or on a local machine and queried locally to
    avoid having to access disk or another machine over a network. 
    Two of the primary concerns when analyzing an AMQ are then, how
    often does it return false-positives and how much space does it require to
    do so?

    The majority of current AMQs offer strong guarantees for individual
    queries; however, the false-positive probability rates for most can be
    pushed toward 1 given the right sequence of non-independent queries.  Some recent
    AMQs have been designed to account for this shortcoming. We call these
    {\bf adaptive} AMQs because they maintain a false-positive probability of
    $\epsilon$ for every lookup regardless of the answers to previous lookups (adapting
    their internal representation whenever a mistake is made so that it will be corrected
    in the future).

    We provide a survey of techniques used for creating adaptive AMQs, comparing and
    contrasting existing implementations, and discussing the tradeoff between practicality
    and ease of analysis.
\end{abstract}

\section{Introduction}

\subfile{sections/introduction}

\section{Various Strategies for Adaptivity}

\subfile{sections/early_attempts_at_adaptivity}

\section{Adaptive Cuckoo Filters}

\subfile{sections/adaptive_cuckoo_filters}

\section{Broom Filters}

\subfile{sections/broom_filters}

\section{Conclusion}

\subfile{sections/conclusion}

\section{Future Work}

\subfile{sections/future_work}

% I setup a bib file
\bibliography{works_cited}
\bibliographystyle{acm}

%\begin{thebibliography}{10}
%    \raggedright
%\end{thebibliography}

\end{document}

