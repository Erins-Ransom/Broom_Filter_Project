\documentclass[../paper.tex]{subfiles}
\begin{document}
Intro paragraph... 

\subsection{Benefits of Rigorous Adaptivity}
While the fully optimized design of the Broom Filter proved difficult to implement,
a naive implementation that uses as much space as it wants would likely be 
more feasible.  This in turn could be used to see the benefits of the Broom Filter's
strong guarantees in actual work loads.  This could also be used to test whether 
strategies that are simpler to implement but may have weaker theoretical 
guarantees, like the Adaptive Cuckoo Filter, actually fair any worse in practice.  

\subsection{A Generalized Framework for Adaptivity}
Bender et al. \cite{broom-filter} made significant strides in showing what is required
for a generalized AMQ to be adaptive, in particular their lower bound on the required 
space, but this can likely be taken even further.  For example, are the discrete 
representations of elements that we see in both the Adaptive Cuckoo and Broom 
Filters also a requirement for adaptivity or could a traditional Bloom Filter  be made 
adaptive using the right strategy?  

\subsection{More Stuff...}


\end{document}

