\documentclass[../paper.tex]{subfiles}
\begin{document}
After the strenuous hardship that is the broom filter implementation attempt,
we present some future directions for this project. We believe that the directions
we pose are actual directions for potential research, not clich\'e generalizations,
and plan to continue exploring these topics.

\subsection{Benefits of Rigorous Adaptivity}
While the fully optimized design of the Broom Filter proved difficult to implement,
a naive implementation that uses as much space as it wants would likely be 
more feasible.  This in turn could be used to see the benefits of the Broom Filter's
strong guarantees in actual work loads.  This could also be used to test whether 
strategies that are simpler to implement but may have weaker theoretical 
guarantees, like the Adaptive Cuckoo Filter, actually fair any worse in practice.  

\subsection{A Generalized Framework for Adaptivity}
Bender et al. \cite{broom-filter} made significant strides in showing what is required
for a generalized AMQ to be adaptive, in particular their lower bound on the required 
space, but this can likely be taken even further.  For example, are the discrete 
representations of elements that we see in both the Adaptive Cuckoo and Broom 
Filters also a requirement for adaptivity or could a traditional Bloom Filter  be made 
adaptive using the right strategy?

\subsection{Making More AMQs Adaptive}
After analyzing the adaptive cuckoo filter \cite{adaptive-cuckoo} and Bender et al.'s work on
making a generalized AMQ adaptive, we aim to apply our newfound knowledge of adaptivity to other
currently non-adaptive AMQs, especially with real world traffic data. For example, we plan to
expand the Bloomier filter \cite{bloomier-filter}, an early attempt at adapting to false positive
using a whitelist, to be adaptive with real world network data by changing the whitelist to a
cache of previously seen false positive.

\subsection{Adaptive Bloom Filters to Speed Up Databases}
Many existing databases, for example Postgres, use bloom filters to speed up disk lookups. In these
applications, many queries are run repeatedly to gather real-time data and it is possible that
some of the queries can repeatedly trigger the same group of false-positives. Is it possible to use
a practical adaptive AMQ, such as the adaptive cuckoo filter, to speed up query lookup times? We can
benchmark this using databases from Harvard courses or on a Postgres fork.

\end{document}

