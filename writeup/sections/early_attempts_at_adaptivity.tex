\documentclass[../paper.tex]{subfiles}
\begin{document}
In the past 15 years, several methods for addressing this particular shortcomings of AMQs have been explored.  
One method seen in the {\it Bloomier Filter} \cite{bloomier-filter} makes use of a whitelist 
W with size $w$ of predetermined potential false-positives.  Whenever there is a lookup 
for an element $x \in W \setminus S$, the Bloomier Filter returns \textit{is false-positive}.   
This works well when the filter can determine problematic queries ahead of time; however, the 
Bloomier Filter requires $O((n+w)\log(1/\epsilon))$ bits of space and the list of bad queries 
cannot be dynamically updated without a potential blow up in space. A more dynamic 
method is to targetedly remove false-positives at the cost of 
introducing random false-negatives.  This can be seen in {\it Retouched Bloom Filters} 
\cite{retouched-filter}; however, not all applications are tolerant of false-negatives 
which makes this option infeasible in many situations.  Yet another method 
that has been used is to assign different numbers of hash functions to different elements
depending on their frequency as seen in {\it Weighted Bloom Filters} \cite{weighted-filter};
however, this technique is dependent on pre-determined statistics gathered about the workload and is not able to dynamically
adapt.  More recently, another method has been explored where the representations for 
elements within the AMQ are changed in response to false positives as seen in the {\it 
Adaptive Cuckoo Filter} \cite{adaptive-cuckoo} and {\it Broom Filter} \cite{broom-filter}. 
Both of these use a technique where a second structure is accessed in tandem 
with the underlying set that is used to provide the information required to adapt to 
false-positives.  We will go into greater detail on these last two, because they are the most
rigorous solutions to adaptive AMQs that we could find, and explore both their theoretical guarantees
and practical concerns.  

\end{document}

