\documentclass[../paper.tex]{subfiles}

\begin{document} Bender et al. \cite{broom-filter} introduce a different adaptive filter, called the ``broom filter''
(because it cleans up after itself). They make two key contributions to the
field of adaptive filters: describing the broom filter data structure itself,
as well as formalizing the idea of adaptivity and providing proofs regarding the guaratees
of the broom filter, and adaptive AMQs in general. The broom filter builds on
a previous AMQ called the quotient filter \cite{quotient-filter}, and, like the adaptive
cuckoo filter, consists of both a local and remote representation. In fact, one
of the main theoretical results they obtain shows that any adaptive AMQ must use
some remote (larger) storage in order to maintain adaptivity. Unfortunately, from
our experience the broom filter is quite impractical. Even a naive implementation
would be cumbersome and time-consuming to write, and important details such as
how adaptivity bits can be stored and maintained efficiently are left out.

\subsection{The Quotient Filter}

The broom filter builds on a previous work called the quotient filter, which
is a non-adaptive AMQ meant to be an optimal and practical alternative to a bloom filter.
The quotient filter uses a single hash function and computes a \textit{fingerprint} for
each element inserted to the set. Given an element $x$, the fingerprint of $x$ is a
prefix of the hash $h(x)$ and is divided into two separate pieces, the \textit{quotient} with
size $q$ bits and the \textit{remainder}, with size $r$ bits. The fingerprint is therefore
the first $q+r$ bits of $h(x)$.
The quotient filter is an array of $2^q$ buckets, where
each bucket consists of $r$ bits, plus three extra metadata bits.
The sizes of the quotient and remainder are tunable
parameters of the quotient filter, which affect the false-positive probability $\epsilon$.

\textbf{Insertions and lookups:} To insert an element $x$ into the quotient filter, we first
compute the fingerprint of $x$: $p(x)$. Then we find the bucket corresponding
to the quotient and insert the remainder bits at that bucket. Assuming the bucket
is empty during insertion, a lookup for $y$ would be simple: we go to the correct bucket as given
by the quotient of $y$ and check if the remainder bits are the same. However, if the bucket is
full, linear probing is used to find an empty bucket, provided two invariants are maintained:

\begin{enumerate}
    \item All remainders with the same quotient are stored contiguously in a \textit{run}, wrapping around to the beginning if necessary.
    \item If remainder $a$ is stored before remainder $b$, then the quotient of $a$ is less than or equal to the quotient of $b$ modulo the wrapping.
\end{enumerate}

We then have to apply the correct metadata bits to all buckets that are touched by the probe. These metadata
bits will allow us to determine which remainders correspond to which quotients,
since due to linear probing remainders will be shifted from their correct quotients.
The three metadata bits provide the following information for each bucket:

\begin{enumerate}
    \item \texttt{occupied} bit: is the bucket occupied?
    \item \texttt{shifted} bit: is the remainder stored at this bucket shifted from its corresponding quotient?
    \item \texttt{run} bit: is this bucket a part of a run?
\end{enumerate}

    Whenever a remainder is
    added, the \texttt{occupied} bit for its intended bucket is always set to 1.  If
    a remainder is not stored in its intended bucket, then the \texttt{shifted} bit
    of the bucket it is stored in is set to 1.  If the remainder has the same
    intended bucket as the remainder immediately before it, the \texttt{run} bit is set to 1.  When we
    perform a lookup for element $x$, we check the bucket indexed by the first
    $q$ bits of $h(x)$.  If \texttt{occupied} is 0, we immediately return
    \textit{absent}; otherwise, if \texttt{shifted} is 0, we scan the current run for a remainder
    that matches the remainder being looked up, and return \textit{present} if a match is found,
    and \textit{absent} otherwise.
    If the \texttt{shifted} bit is 1, then we must
    first find the beginning of the {\bf cluster}, a group of runs with no
    empty buckets between them, and then scan forward keeping track of the
    number of runs and the number of occupied buckets to determine the location
    of our target run.  If we reach the end of the cluster without finding it,
    then we return $false$, but if we do find the run, we compare to the
    remainders in that run and return $true$ or $false$ accordingly.  Deletes
    can be performed by looking up the corresponding remainder (of which their
    may be multiple) and removing it and then shifting other remainders and
    updating metadata bits as needed.
    
    Let $k$ be the number of occupied buckets in the filter. A false positive thus occurs if $h(x) =
    h(y)$ when $x \in S$ and $y \notin S$.  Assuming $h: {\bf U} \rightarrow
    \{0, ... , 2^{q+r}-1\}$ generates outputs that are distributed uniformly
    and independently, then the probability of a false positive is given by $$
    1 - \big(1 - \frac{1}{2^{q+r}}\big)^k \approx 1- e^{-k/(q+r)} \leq
    \frac{k}{2^{(q+p)}} \leq \frac{2^q}{2^{(q+r)}} \leq 2^{-r}.$$

    If we would like to choose $r$ and $q$ such that the filter has a false-positive
    probability of at most $\epsilon$, and $n$ is the maximum set size, we can choose $q=\log(n)$
    and $r=\log(1/\epsilon)$.
    With these choices, the quotient filter achieves a false-positive probability of at most
    $2^{-r} = \epsilon$ while using $O(2^q r) = O(n \log (1/\epsilon)$ space.  	

\subsection{Adding Adaptivity} Now that we have a better understanding of the
original data structure, we will now discuss how it was modified to achieve
adaptivity.  As mentioned before, one difference between the Broom Filter and
the Quotient Filter is that the Broom Filter has both a local and remote
representation.  The local is a fully functioning AMQ on its own, but the
remote gives the local additional information which is required for it to adapt
to false positives.  Specifically, whenever there is a false-positive caused by
some element $x \notin S$, the job of the remote is to provide the element $y
\in S$ that caused the false-positive, this way the fingerprint for $y$ can be
modified in the local representation so that future lookups for $x$ will no
longer result in false positives.  The specific implementation of the remote is
not detailed other than that it can be done using standard data structures.
Needing to access a remote portion of the filter may seem counter intuitive
given that AMQs are designed to avoid remote lookups, but the cost of accessing
the remote portion is amortized by the fact that it is only ever accessed when
the underlying dictionary already needs to be accessed: for inserts/deletes and
whenever there is a false-positive.  Due to this, the accesses to the remote
portion are essentially free asymptotically.
	
    Now to how the local representation differs.  There are actually two
    different designs for the local representation depending on the size of the
    remainders, $\log(1/\epsilon)$, both of which make use of two levels.   In
    the {\bf small remainder} case were $r \leq 2 \log\log n$, both levels
    resemble Quotient filters with independent hash functions.  In the {\bf
    large remainder} case, were $ r > 2 \log\log n$, then only a single hash
    function is used and the relationship between the two levels has to do with
    backyard hashing \cite{backyard-hashing}.  Most of these changes from the
    original Quotient Filter are in order to improve lookup and insert
    performance, so we will not go into as much detail on them and instead
    focus on the changes that enable adaptivity.  In both cases, this is
    achieved through the use of {\bf adaptivity bits} which are stored
    separately from the rest of the structure.  Both also make use of hash
    functions of the type $h : {\bf U} \rightarrow \{0, ..., n^c\}$ for some
    constant $c \geq 4$.  In the small remainder case, these are maintained
    separately for each level, and in the large remainder case they are
    maintained for the whole structure.  Suffice to say, both representations
    require $O(n\log(1/\epsilon)$ space as well as $O(n)$ space for the
    adaptivity bits.  
	
    What the adaptivity bits are is an extention of the {\bf base fingerprint}
    which is comprised of the first $q + r$ bits of the hash.  So if a
    fingerprint has $a$ adaptivity bits, the fingerprint is a prefix of the
    respective hash with length $q + r + a$. The adaptivity bits do not have a
    fixed length and may vary between fingerprints while some fingerprints may
    not have any.  Adaptivity bits may be added to a fingerprint in two cases:
    {\bf 1)} when a fingerprint is added to the Broom Filter and {\bf 2)} when
    a false-positive occurs.   When adding fingerprints to the filter, the
    invariant that no fingerprint is a prefix on another is always maintained.
    In order to maintain this, a newly added fingerprint must be checked
    against any others with the same quotient.  If any of those also share the
    same remainder as well, then adaptivity bits are added to both until the
    invariant is once again true.  Since the invariant is maintained in this
    way, the fingerprint being added can only ever be a prefix of at most one
    other fingerprint in the filter.   While the hash for the new fingerprint
    may be readily available, a call to the remote representation is required
    to obtain the hash for the offending fingerprint already in the table
    (assuming there is one).  However, since we are already accessing the
    underlying dictionary to insert a new element to it, the access to the
    remote representation is permissible.  Additional adabivity bits may be
    added at this step due to how deletes are handled, but we will discuss that
    later.  When a false-positive occurs for a lookup of $x \notin S$, we again
    must access the remote but this time to find the hash of the offending
    element $y \in X$ whose fingerprint was a prefix of $h(x)$.  Once this is
    obtained from the remote, adaptivity bits are added to the fingerprint of
    $y$ in the local representation until the fingerprint of $y$ is no longer a
    prefix of $h(x)$.  Due to the invariant, there can only be one such $y$
    that needs to be extended. 
	
     If we ignore deletes, then what we have already described is sufficient to
     achieve adaptivity since no a lookup for any given element will never
     result in a false positive more than once.  That said, the Broom filter
     does support deletes and we should also be concerned about the continued
     addition of adaptivity bits to the local representation.  Before we
     address these points, it is worth mentioning that the maintenance of this
     invariant and the adaptivity of the filter is dependent on the hash
     function being one-to-one.  Consider if $x \notin S$, $y \in S$ and  $h(x)
     = h(y)$.  In such a case, adding adaptivity bits to the fingerprint of $y$
     will never make it not a prefix of $h(x)$.  This could potentially be
     resolved by allowing fingerprints to grow beyond the size of their hash,
     but this would likely have problematic implications for the size of the
     local representation.  This does not appear to be addressed in the design
     of the Broom Filter.  Deletes in the Broom Filter are carried out like
     those in a Quotient Filter except that the quotient and the adaptivity
     bits are not removed but are left as a {\bf ghost} in the filter.  When a
     new fingerprint is added, if there is a matching ghost in the filter, then
     the new fingerprint takes on all the adaptivity bits of the ghost even  if
     this is more than required for the standard insert process.  What this
     achieves is that false positives can no longer be generated by timing
     attacks involving repeatedly adding and removing an element from the
     filter that collides with some element not in the filter.  This is
     because, a re-inserted element essentially retains its adaptivity bits
     from before it was removed and thus will not collide with any elements
     that collided with it in the past.   
	 
     Finally, we must address the issue of continually growing adaptivity bits.
     This is done with a deamortized equivalent of rebuilding the the filter
     every $\Theta (n)$ times adaptivity bits are added.  This is done by
     keeping two hash functions, $h_a$ and $h_b$, and having phases that
     gradually switch between hash functions.  At the beginning of a phase,
     frontier $z = -\infty$ and only a $h_a$ is used.  Once adaptivity bits are
     added, the smallest constant $k > 1$ elements of $S$ that are greater than
     $z$ are deleted from the filter (including their ghosts) and re-inserted
     using $h_b$, after which $z$ is set to be the largest element that was
     re-inserted.  This requires a access to the remote representation, but
     again this will only happen when the underlying dictionary is already
     being accessed.  When an element is $x$ looked up or inserted, $h_a$ is
     used if $x > z$ and $h_b$ is used otherwise.  Once $z$ reaches the largest
     element in $S$, then a new phase begins.  This is then enough to guarantee
     that, with high probability, there are $O(n)$ adaptivity bits in the
     filter at any given time.  The Broom Filter thus achieves adaptivity while
     still only using $O(n \log (1/\epsilon)$ space locally.  


\subsection{Implementation details}

Bender et al. \cite{broom-filter} provide a thorough description of the theory of
the broom filter, but some key implementation details are left out.
Most notably the storage of the adaptivity bits is a real problem
in practice because they are variable sized chunks of memory that
are often smaller than a machine word. This means that in practice,
to store the adaptivity bits, we also need to keep track of sizes
for each chunk, and need to dynamically resize when more bits are needed.
This is a lot of overhead, especially in real applications the adaptivity
bits don't use much space in the first place (a universe of 64-bit integers,
which is common for database join operations). Where to store the adaptivity
bits is also unclear. We need adaptivity bits for every element in
the set, but since our set is stored as a quotient filter, unless
we want to be constantly resizing the entire quotient filter, the
adaptivity bits should be stored in a separate memory. In general
the concept of adaptivity bits is impractical because computers perform
well when working with constant, repetitive data, which is the opposite
of what is needed to implement adaptivity bits.

Another key detail that proves impractical is the RevLookup operation
needed in the remote data structure for extending the adaptivity bits. When
an insertion happens, there may be fingerprints stored in the broom filter
which are prefixes of the item being inserted. In that case the adaptivity
bits of the colliding element need to be extended until they are no longer
a prefix of the hash being inserted. To extend the adaptivity bits, we
need the full hash of the colliding element, but since we only have its
fingerprint stored in the broom filter, this will require a lookup in
the remote storage. Efficiently supporting this lookup requires additionally
maintaing a table of fingerprints to full hashes in the remote data structure
which is quite costly because every time a fingerprint is extended in the
broom filter, the same fingerprint needs to be extended in the remote table
(which will require re-hashing the fingerprint if the remote table is maintained
as a hash table).

Other difficulties with implementation arise with the hash function. \cite{broom-filter}
mention that the hash function must map $U \rightarrow \{0,\ldots,n^c\}$ where
$n$ is the maximum set size and $c$ is a constant greater than 3. If the universe
is larger than $n^c$ then it is possible to have a complete collision between
$h(x)$ and $h(y)$. This has a low probability of occurring but if it does
occur and the colliding element is repeatedly queried, then the false positive
rate will be close to 1, causing the adaptivity of the filter to fall apart. Therefore
our hash function must hash from the universe to something larger than the universe
and be a perfect hash function to avoid these collisions. This immediately rules
out using the broom filter for strings because in that case the universe is too large.
In addition, with a universe of 64-bit integers, we have to use a hash function
that maps to integers larger than 64 bits (unless we want a maximum set size capped
at $2^{16}$ which is fairly small), and maintaining integers larger than the computer's
word size is impractical.

\subsection{A Lower Bound for Adaptivity} While we may have found the
implementation of the Broom Filter to be somewhat unwieldy in practice, there
is still a very important takeaway.  As part of the Broom Filter's design,
Bender et al. also prove that any AMQ storing a set of size $n$ from a universe
of size $u > n^4$ requires $\Omega (\min \{n\log n, n\log\log u\})$ bits of
space with high probability in order to maintain a sustained false-positive
rate $\epsilon < 1$.  This is why the remote representation is required if the
local representation is to be near the optimal size for a traditional AMQ and
why it is not surprising that the Adaptive Cuckoo Filter also shares this two
part structure.  A similar proof was explored with the {\it Bloomier Filter} 
\cite{bloomier-filter} in assessing the space required when dynamically 
updating it's whitelist. The proof makes use of adversarial model where the 
goal of the adversary is to adaptively generate a sequence of $O(n)$ lookups 
that force the AMQ to either use too much space or to fail to sustain its 
false-positive rate of $\epsilon$.  
	
	{\bf The Adversary's Attack:} The adversary starts with a set $S$ of size $n$ chosen
	uniformly at random form $U$.  The attack then proceeds in rounds.  For each round, 
	a set $Q$ of size $n$ is chosen uniformly at random from $U\setminus S$.  All members
	of $Q$ are then looked up. On each subsequent round, any false-positives from the 
	previous round are queried again. 
	
	The general idea of the proof is that, we expect the adversary to find a certain 
	fraction of false-positives which is shown to have a high concentration using Chernoff
	bounds.  Therefore, with high probability, the AMQ will have to fix an $\epsilon$ fraction
	of false-positives in each round or fail to sustain it's false-positive rate.  Each time the 
	AMQ corrects a false-positive it must change its configuration.  Each configuration of the 
	AMQ for a given set $S$ can be defined using the list of false-positives that have been
	corrected.  This results in a one-to-one mapping between the set of fixed false-positives 
	and the configurations of the AMQ.  The number of configurations can then be lower 
	bounded by the number of sets of false-positives that the adversary can force the AMQ
	to fix.  The space required for the AMQ is thus lower bounded by the amount of space 
	required to enumerate its configurations.  Finally, the bound is shown to be tight by 
	providing a construction on an AMQ, the Broom Filter, that uses $O(\min \{n\log n, 
	n\log\log u\})$ bits of space with high probability. 
	
	


\end{document}

