\documentclass[../paper.tex]{subfiles}
\begin{document}
\newcommand{\on}{\operatorname}
The \emph{adaptive cuckoo filter} \cite{adaptive-cuckoo} is a practical solution to the AMQ problem with adaptivity, combining a cuckoo filter and cuckoo hash table. Both of these structures are based on the
technique of \emph{cuckoo hashing}, where having two choices for each object allows for constant lookup times. We now describe the tools used in an adaptive cuckoo filter.

\subsection{Cuckoo Hashing}
The adaptive cuckoo filter uses a cuckoo hash table \cite{cuckoo-hash} and the cuckoo filter \cite{cuckoo-filter}. As the name suggests, the cuckoo filter is an
efficient data structure for AMQ's. Both objects maintain an array of length $n$. Each entry of the array is known as a \emph{bucket}, and a bucket can have either one or multiple \emph{slots}.
The structures use two hash functions $h_1,h_2:U\to \{0,\ldots,n-1\}$ to determine where each element of the set is stored.

In a cuckoo hash table, the entire key is stored in the bucket. In a cuckoo filter, only a fingerprint is stored. Insertions, lookup and deletes are performed as follows. For convenience, let $x$ be the argument in all three procedures.

\textbf{Insert:} Compute both $h_1(x)$ and $h_2(x)$. If either of the two array entries (buckets) indexed by the $h_1(x)$ and $h_2(x)$ has remaining slots, we insert $x$ into that slot.
Otherwise, we pick one of the occupied slots at random. Suppose that slot currently stores the item $y$. Then, we evict $y$, insert $x$ into the now-empty slot and perform \texttt{INSERT(y)}.

This process will succeed as long as we do not run into cycles, which occurs with low probability as long as the load is bounded. Another issue is that we need to compute $h_1(y),h_2(y)$
when we evict $y$ and insert it back. This is not possible in a normal cuckoo filter since we only have access to the fingerprint, and the issue is resolved using a technique known as \emph{partial-key cuckoo hashing}.
However, this is not necessary when a cuckoo hash table and filter work in tandem, as we will explore in the next section.

\textbf{Lookup:} Compute $h_1(x)$ and $h_2(x)$ and check if either of the two buckets stores item $x$ (or its fingerprint). If it does, return the object (or \textit{present}, for a cuckoo filter). Otherwise, return \texttt{absent}.

\textbf{Delete:} After looking up the location of $x$, remove the item (or its fingerprint) from the data structure.

The cuckoo hash table provides worst case $O(1)$ lookup time and amortized $O(1)$ insertion time, as long as the load factor, the proportion of slots that are occupied, is less than some constant $c$.
The value of $c$ is around $50\%$ when the bucket can only store 1 item, and increases to over $90\%$ when the bucket can store 3 items. The worst case $O(1)$ lookup is a significant improvement over hash tables
with other strategies of resolving collisions such as linear probing. The general phenomenon that having two choices increases a data structure's performance is named by Mitzenmacher as ``the Power of Two Choices'' \cite{prob-textbook}.

In the cuckoo filter, a lookup causes a false positive only if two elements share the same hash and the same fingerprint. Then, we can tune the false positive probability $\epsilon$ using extra space by increasing the length of the fingerprint.
The filter is more space efficient than the Bloom filter when the false-positive rate $\epsilon$ is relatively small (less than $3\%$), and continues to perform even when the load is above $90\%$.
Fan et. al \cite{cuckoo-filter} generalized the filter to include multiple slots per bucket and used a technique known as \emph{semi-sorting} to further improve the filter's space efficiency when buckets have more than 1 slots.

\subsection{Adaptive Cuckoo Filter}
Mitzenmacher et.\ al introduced in \cite{adaptive-cuckoo} the adaptive cuckoo filter, which can answer approximate membership queries while adapting to false positives. We will use a framework similar
to the one used in the broom filter analysis \cite{broom-filter} to facilitate comparisons and proofs in the future.

The adaptive cuckoo filter uses a cuckoo hash table and cuckoo filter, as described above. The cuckoo hash table would be stored remotely since it is only used when adapting to false positives.
Instead of storing on a fingerprint $p$ in the cuckoo filter, we store a pair $(p,\alpha)$, where $p$ is the fingerprint and $\alpha$ is a number with $s$ bits that will help the filter adapt. 
The filter maintains a family of hash functions $f_0,\ldots,f_{2^s-1}$. The fingerprint of an element $x\in U$ and a number $\alpha$ is equal to $f_\alpha(x)$.

The filter and the table satisfy the following invariant: for each element $x$ in the hash table, $x$'s fingerprint is stored in the same location in the Cuckoo filter as it is in the table.

\textbf{Insertions and Deletions:} To insert or delete an element $x\in S$, perform the operation in both the cuckoo hash table and the cuckoo filter.
Since the two structures use the same hash function, the element will be inserted into the same location in the array. Therefore, the invariant is maintained.
The cuckoo hash table inserts element $x$ and the cuckoo filter inserts the pair $(f_0(x),0)$.

When we evict a finger $p_y$ from the cuckoo filter, the corresponding element $y$ is also evicted from the cuckoo hash table.
Since the operation already accesses the remote hash table, retrieving the element $y$ from the hash table has minimal overhead. As a result, partial-key cuckoo hashing is not necessary.

\textbf{Lookup and Adapting:} To check if an element $x$ is in the set, we first calculate $h_1(x), h_2(x)$. If one of the bucket contains a pair $(p,\alpha)$ satisfying $f_\alpha(x)=p$, i.e. the fingerprint matches, return \textit{present}.

A false positive occurs when an element $y\not\in S$ shares the same fingerprint as $x\in S$. This is detected at the time when we retrieve the element from the
Cuckoo hash table. To adapt for this false positive, we increment $\alpha$ and change the fingerprint for $x$ accordingly. It is unlikely for $x,y$ to share the same fingerprint again.
Therefore, the filter has adapted to this false positive. The adaptive cuckoo filter shares a similar generalization as the cuckoo filter for cases where each bucket has more than one cell, with some changes to the adapting algorithm.

Although their paper does not present many theoretical guarantees about their filter, Mitzenmacher et.\ al performed simulations with randomly generated queries and real-world data and found that in practice the adaptive cuckoo filter beats the cuckoo filter in most configurations where some queries are repeated more than others. Cuckoo filters already boast high efficiency in practice, which makes the adaptive cuckoo filters even more impressive.

The adaptive cuckoo filter is also easy to implement because each bucket uses a constant number of adaptivity bits that does not change over time.
As a result, the memory for the adaptivity bits can be allocated at the start and it is easy to keep them compact. We will see that this is not the case
for the broom filter and that this makes an efficient broom filter implementation almost impossible.
\end{document}
