\documentclass[../paper.tex]{subfiles}
\begin{document}

    Bloom filters and other approximate membership query data structures, or
    AMQs, are often used to help avoid costly lookups to large dictionaries.
    They achieve this by keeping a compact representation of a set
    $S$ of keys from a universe $U$.  All AMQs support inserts and lookups
    while some also support deletes.  When performing a lookup for an item $x$
    that has been inserted, i.e. $x\in S$, the AMQ is guaranteed to return
    {\it present}.  The downside of their compact representation of $S$ is
    that, when a lookup is performed for an item $x \notinS$, {\it present}
    may be returned with probability $\epsilon$.  This {\bf false-positive
    probability} is a usually a tunable parameter with a lower $\epsilon$
    typically requiring more space to be used by the AMQ.  That said, their
    compact representation is also what makes them helpful since they can be
    kept in memory or on a local machine and queried locally to avoid having to
    access disk or another machine over a network.  As such, two of the primary
    concerns when analyzing an AMQ are then, how often does it return
    false-positives and how much space does it require to do so?

    The majority of current AMQs only offer strong guarantees for independent
    queries, but the false-positive probability rate for most of these can be
    pushed toward 1 given the right sequence of queries.  In the adversarial
    case, this can be done by repeated lookups of elements that result in
    false-positives, but, even in regular workloads, the lack of strong
    guarantees for sequences of lookups can lead to poor performance depending
    on the circumstances.  For example, Mitzenmacher et al.
    \cite{adaptive-cuckoo} discuss how repeated lookups can arise during packet
    processing because the lookups may correspond to flow identifiers.  Their
    solution to this problem was to create a variant of the Cuckoo Filter
    \cite{cuckoo-filter} which adapts to false positives by removing them for
    future queries and show using simulations that this does indeed result in a
    better false positive rate under such circumstances.  Bender et al.
    \cite{broom-filter} take a similar approach in designing the Broom Filter
    which is a variant of the Quotient Filter \cite{quotient-filter} that also
    adapts to false positives, but rather than showing their advantages through
    simulations, they define an {\bf adaptive} AMQ as one that maintains a
    false-positive probability of $\epsilon$ for each query regardless of the
    answers to previous queries and show that the Broom Filter is proveably
    adaptive.

\end{document}
