\documentclass[../paper.tex]{subfiles}
\begin{document}
After having discussed the designs of both the Adaptive Cuckoo Filter and the 
Broom Filter, there seems to be a general strategy that can be applied to 
achieve a fairly strong amount of adaptivity in an AMQ without a prohibitive 
increase in the amount of local space used.  We start by taking a traditional 
AMQ that uses discrete representations of its members, like the fingerprints 
used in the Quotient and Cuckoo filters, so that we have the ability to 
change the representations of individual elements in the filter and use that as 
our local representation.  We then add a second structure which is updated and 
accessed in tandem with underlying set.  This second structure then needs to be 
able to supply sufficient information for the local representation to change the 
representations of individual elements in order to correct for false-positives.  As 
shown by Bender et al.'s proof \cite{broom-filter}, this second structure is 
critical to keeping the size of the local representation near optimal.  

	As shown by the Broom Filter, this strategy can be taken to the extreme where the 
	method of adaptation goes as far as to guarantee that two elements will never 
	collide more than once within a given phase.  Alternatively, we can use a slightly 
	more relaxed method of adaptation as seen in the Adaptive Cuckoo Filter where, 
	once two elements have collided, they are very unlikely to collide again.  After 
	having attempted to implement the rigorous method of the Broom Filter, we have
	an increased appreciation for the simplicity of Adaptive Cuckoo Filter's design, 
	even if it does not lend itself as nicely to rigorous analysis.  
	
	It is also worth noting that this strategy for adding adaptivity does not lend itself 
	well to all traditional AMQs.  For example, consider a traditional Bloom Filter where 
	each element in the filter is represented by $k$ bits in a table that are assigned by 
	$k$ hash functions.  Because the representations of multiple elements can overlap, 
	updating a single one without changing the others becomes problematic and 
	potentially prohibitive to adaptivity.  That said, filters that make use of discrete 
	representations for each element tend to make better use of space in the first 
	place through methods like quotienting in the Quotient and Broom Filters.  

\end{document}

